\chapter{Математические модели транспортных потоков}\label{ch:ch1}

\section{Макроскопические модели}\label{sec:ch1/sec1}

Развитие макроскопического моделирования транспортных потоков восходит к исследованиям движения жидкости и газа в середине прошлого века. 
В этом разделе мы рассмотрим основные модели предложенные в то время, их связь с современными исследованиями, а также возможные обобщения.

\subsection{Модель Лайтхилла-Уизема-Ричардса (LWR)}\label{subsec:ch1/sec1/sub1}

В 1955 году независимо в работах~\cite{LWR/lighthill1955kinematic,richards1956shock} была по видимому предложена первая гидродинамическая модель однополосного транспортного потока названная в честь её создателей моделью Лайтхилла-Уизема-Ричардса (LWR).
Однополосная дорога в данном случае считалось бесконечной в обе стороны дорогой с движением слева на право без источников и стоков автомобилей.
Поток автомобилей рассматривался как поток сжимаемой жидкости в рамках следующих постулатов:
\begin{itemize}
  \item существует взаимно однозначное соответствие между скоростью \(v(t, x)\) и плотностью потока АТС \(\rho(t, x)\) называемое уравнением состояния;
  \item выполняется закон сохранения числа АТС.
\end{itemize}
Где \(t\)~--- момент времени, \(x\)~--- координата трассы.
Таким образом запись \(\rho(t, x)\), \(v(t, x)\) обозначает соответствующую величину в момент времени \(t\) в окрестности координаты \(x\).
В дальнейшем в рамках макроскопического подхода мы предполагаем, что транспортный поток подчиняется некоторой микроскопической модели, в которой детально описывается поведение АТС в зависимости от окружающей обстановки и эта модель является разностным или дифференциально-разностным аналого рассматриваемой нами макроскопической модели.
Рассматривание же именно макроскопической модели обусловлено как большей наглядностью в виде гидродинамических аналогий так и ввиду относительной простоты их исследования.

Итак, первое предположение можно выразить следующим образом:
\begin{equation}
    \label{eq:equation1}
    \left\{
    \begin{array}{rl}
        v(t, x) = V(\rho(t, x)),\\
        V'(\rho(t, x)) < 0
    \end{array}
    \right.  
\end{equation}
Предполагается, что скорость не может возрастать с увеличением плотности потока АТС.
Обозначим
\[
    Q(\rho) = \rho V(\rho)
\]
количество АТС проходящих в единицу времени через заданное сечение.
Зависимость \( Q(\rho) \) называют также фундаментальной диаграммой потока.
Понятие фундаментальной диаграммы, её построение на реальных данных и ограничения на неё накладываемые являются одной из тем многих статей макроскопического моделирования.
Отметим только, что понятие фундаментальной диаграммы не совсем корректно на малых плотностях АТС \( \rho \backsim 50-120 \) АТС/км~\cite{kerner2009introduction}. 
Иначе говоря на таких плотностях нет чёткой зависимости скорости от плотности.

Закон сохранения выражается следующим образом:
\[
    \int_{a}^{b} \rho(t + \delta, x)dx + \int_{a}^{b} \rho(t, x)dx = -\left\{\int_{t}^{t + \delta} Q(\rho(\tau, b))d\tau - \int_{t}^{t + \delta} Q(\rho(\tau, a))d\tau \right\}.
\]

Таким образом любого прямоугольного контура при $x\in \mathbb{R}, t\geq 0$ выполняется:
\[
    \oint \rho(t, x)dx - Q(\rho(t, x))dt = 0.
\]
Отсюда, для любого кусочно гладкого контура в точках гладкости \( \rho(t, x) \):
\begin{equation}
    \label{eq:equation2}
    \frac{\partial \rho}{\partial t} + \frac{\partial Q(\rho)}{\partial x} = 0.
\end{equation}

Дополним это уравнение начальными условиями типа Римана
\begin{equation}
    \label{eq:equation3}
    \rho(0, x) = 
    \left\{
    \begin{alignedat}{3}
        &&\rho_{-},\quad &x<x_{-},  \\
        &&\rho_{0}(x),\quad x_{-}\leq &x<x_{+},  \\
        &&\rho_{-},\quad &x\geq x_{+}.
    \end{alignedat}
    \right.    
\end{equation}

В дальнейшем соотношения~\cref{eq:equation1,eq:equation2} вкупе с различными граничными условиями~\cref{eq:equation3} и фундаментальной диаграммой потока позволяют решать различные задачи от моделирования движения автомобилей на свободной трассе до задач распространения затора, в которой одно из граничных условий приобретает вид \(\rho_{+} = \rho_{max}\) --- то есть плотность автомобилей максимальна.


\subsection{Модель Танака}\label{subsec:ch1/sec1/sub2}
В 1963 году Танака и другие~\cite{inose1983control} предложили один из способов определения зависимости \( V(\rho) \).
В условиях однополосного потока и ограниченности максимальной скорости сверху предлагалось:
\[
\rho(v) = \frac{1}{d(v)},
\]
где 
\begin{equation}
    \label{eq:equation4}
    d(v) = L + c_1v + c_2v^2
\end{equation}
--- среднее расстояние между АТС при заданной скорости движения. 
Величину $d(v)$ также называют динамическими габаритами.
Её также можно интерпретировать как участок автомобиля вместе с его дистанцией экстренного торможения.

В формуле~\cref{eq:equation4} \(L\)~--- средняя длина автомобилей, \(c_1\)~--- характеристика реакции водителей, \(c_2\)~--- коэффициент тормозного пути, зависящий в первую очередь от погодных условий.

Несмотря на свою простоту модель Танака играет важную роль в исследованиях транспортных потоков до сих пор~\cite{gartner2002traffic}, а динамические габариты автомобилей использованы и в данной работе.

\subsection{Модель Уизема}\label{subsec:ch1/sec1/sub3}
Следующий шаг упоминающийся еще в середине прошлого века, но окончательно предложенный лишь в 1974 году Дж. Уиземом~\cite{uisem1977linear} заключается в учёте того факта, что автомобилисты принимают решение о увеличении и уменьшении своей скорости на основе плотности потока перед ними, что выливается в следующее изменение~\ref{eq:equation1}:
\[
    v(t, x) = V(\rho(t, x)) - \frac{D(\rho(t, x))}{\rho(t,x)}\frac{\partial\rho(t,x)}{\partial x},\ \ D(\rho) > 0.
\]
Откуда, с учетом закона сохранения числа АТС~\ref{eq:equation2} получаем закон сохранения с нелинейной дивергентной диффузией:
\[
    \frac{\partial \rho}{\partial t} + \frac{\partial Q(\rho)}{\partial x} = \frac{\partial}{\partial x} \left(D(\rho) \frac{\partial \rho}{\partial x}\right).
\]
Таким образом получаем уравнения учитывающие тот факт, что автомобилисты снижают скорость при увеличении плотности АТС впереди и наоборот.


\subsection{Модель Пейна}\label{subsec:ch1/sec1/sub4}
Модель Пейна предложенная в 1971 году~\cite{payne1971model} также является учётом особенностей реального движения в транспортной сети.
В данной модели закон сохранения числа автомобилей все также имеет вид~\cref{eq:equation2}
\[
    \frac{\partial \rho}{\partial t} + \frac{\partial \rho v}{\partial x} = 0.
\]

Теперь, однако, учитывается тот факт, что при изменении плотности АТС на трассе скорость не может моментально изменится, а меняется следующим образом:
\[
\frac{d}{dt}v = \frac{\partial v}{\partial t} + v \frac{\partial v}{\partial x} = -\frac{1}{\tau} \left(v - \left(V(\rho) - \frac{D(\rho)}{\rho}\frac{\partial\rho}{\partial x}\right)\right),
\]
где скорость стремиться к некоторой желаемой скорости \( V(\rho) - \frac{D(\rho)}{\rho}\frac{\partial\rho}{\partial x}\) с некоторой характеристикой скорости стремления \(\tau\).
Отметим, что в литературе принято называть моделью Пейна частный случай вышеизложенной модели с \(D(\rho)\equiv\tau c_0^2 > 0\)

Перепишем систему уравнений в следующем виде:
\begin{equation}
    \label{eq:equation5}
  \frac{\partial}{\partial t}\binom{\rho}{v} + 
  \left(
        \begin{array}{cc}
            v  & \rho \\
            D/(\tau\rho)  & v \\
        \end{array}
    \right)\cdot \frac{\partial}{\partial x} \binom{\rho}{v} = \frac{1}{\tau}\binom{0}{V-v}.
\end{equation}
Поскольку матрица при \(\frac{\partial}{\partial x}\) имеет различные собственные вещественные значения данная система очевидно является гиперболической.


\subsection{Современные исследования и обобщения}\label{subsec:ch1/sec1/sub5}
В дальнейшем развитие макроскопических моделей привело к нелинейным гиперболическим уравнениям второго порядка для скорости и плотности потока в различны постановках, зависящих от конкретного метода учёта тех или иных особенностей поведения автомобилистов на автомагистрали~\cite{payne1971model,daganzo1995requiem,papageorgiou1998some,zhang2002non,zhang2003anisotropic,siebel2006fundamental}.
Однако, в~\cite{collectiveArticle} было показано что все они так или иначе сводятся к следующему диагональному виду:
\[
\left\{
\begin{array}{rl}
  &\frac{\partial\rho}{\partial t} + \frac{\partial Q}{\partial\rho}\frac{\partial\rho}{\partial x} = f_0 \\
  &\frac{\partial v}{\partial t} + \frac{\partial Q}{\partial\rho}\frac{\partial v}{\partial x} = (\frac{\partial Q}{\partial\rho} - v) \frac{f_0}{\rho} + f_1
\end{array}
\right.
\]
путём использования обобщенного уравнения состояния --- зависимости давления от плотности транспортного потока \(P(\rho) = \int_{0}^{\rho} c(\tilde{\rho})^2 d\tilde{\rho} = \int_{0}^{\rho} \tilde{\rho}^2 (\frac{\partial v(\tilde{\rho})}{\partial \tilde{\rho}})^2\), 
замыкающего исходную систему уравнений.
При этом \(f_1\) в правой части будет играть роль релаксационного слагаемого в случае необходимости в нём.

Окончательно, в~\cite{collectiveArticle} было показано что именно вид уравнения состояния, замыкающего систему модельных уравнения --- зависимости интенсивности транспортного потока от его плотности, полностью определяет свойства любой феноменологической модели.

Отметим, что для нелинейного закона сохранения~\ref{eq:equation2} гладкое решение задачи Коши~\ref{eq:equation2, eq:equation3} существует только в малой окрестности линии начальных условий. Для нелинейных уравнений с разрывными начальными условиями решение задачи Коши не определяется даже в сколь угодно малой окрестности начальных условий.
Для разрешимости поставленной задачи с такими начальными условиями необходимо рассматривать разрывные решения уравнения и иначе ставить задачу Коши.
Таким образом для различных постановок из разделов~\ref{subsec:ch1/sec1/sub1,subsec:ch1/sec1/sub2,subsec:ch1/sec1/sub3,subsec:ch1/sec1/sub4} и др. необходимо отдельно исследовать разрешимость задачи в случае разрывных начальных условий.
Это накладывает существенные ограничения на моделирование светофорного управления автомагистралью когда светофор резко перекрывает поток автомобилей, то есть разрывным образом изменяет плотность автомобилей и их скорость и вынуждает исследователя проводить дополнительные доработки и модификации для проведения такого рода экспериментов.


\FloatBarrier