
Работа посвящена мезоскопическому моделированию транспортных потоков в большой 
автомобильной сети с использование данных из разнородных источников и групп 
автомобильно-транспортных средств (АТС).

{\actuality} Развитие современных технологий сбора и хранения данных, а также
удешевление аппаратуры по видеофиксации и развитие технологий автоматизированного 
обнаружения нарушений правил дорожного движения (ПДД) вкупе со сбором больших объемов 
данных о движении транспортных средств по автомобильно-транспортным сетям путём отслеживания 
GPS и ГЛОНАСС навигаторами привело к значительному увеличению объёмов разнородных данных о 
движении автомобилей в транспортных сетях.
Данная ситуация не только открыла для исследователей возможности по моделированию
более крупных транспортных сетей, но и поставила новые задачи связанные с вычислительной
сложностью решения такого рода задач.

Классические работы по моделированию транспортных потоков восходят к 50-ым годам прошлого века, когда наблюдалось бурное развитие газовой динамики.
Тогда же появились первые макроскопические модели, в которых транспортный поток уподобляется потоку «мотивированной» сжимаемой жидкости.
В модели Лайтхилла – Уизема – Ричардса (LWR)\autocite{LWR/lighthill1955kinematic} транспортный поток уподобляется птоку сжимаемой жидкости и 
описывается законом сохранения количества автомобилей.
При этом в модели постулируется существование однозначной функциональной зависимости между величиной интенсивности транспортного потока автомобилей и его плотностью. 
Эту зависимость называют фундаментальной диаграммой.
В современном же макроскопическом подходе транспортный поток описывается нелинейной системой гиперболических уравнений второго порядка для плотности и скорости потока в различных постановках~\autocite{siebel2006fundamental, collectiveArticle}.

Другим подходом к моделированию транспортных потоков является микроскопический подход в котором моделируется движение каждого транспортного средства в системе.
Классическим примером такой модели является модель следования за лидером, где скорость каждого автомобиля рассчитывается в зависимости от характеристик впередиидущего АТС (лидера)~\autocite{gasn2017introd}.
В современных исследованиях также пытаются учесть разнородность транспортных стредств в потоке АТС.
В работе ~\autocite{dey2008simulation} детально рассматривается движение транспортного потока состоящего из автомобилей, автобусов, двухколесных и трехколесных мотоциклов на двухполосной дороге.
В ~\autocite{guo2011dynamics} рассматривается смешанный поток из велосипедов и автомобилей.
В ~\autocite{gundaliya2008heterogeneous, lan2005inhomogeneous} для той же задачи моделирования смешанного потока используются клеточные автоматы.

Работ по мезоскопическому подходу в моделировании транспортных потоков достаточно мало и они достаточно различны чтобы выделить какое либо направление развития в них.
Из относительно современных подходов в мезоскопическом моделировании можно выделить~\autocite{oskarbski2018applying}, где рассматривается комбинация микро-, мезо- и макроскопических моделей для расчета выделения углекислого газа в атмосферу в транспортной сети, а также~\autocite{tordeux2018mesoscopic}, где мезоскопическая модель используется для моделирования пешеходного движения, однако в ней проводятся расчеты вычислительной сложности полученной модели и проводится сравнение зависимости вычислительных затрат относительно плотности потока пешеходов для рассматриваемой мезоскопической модели и выбранных микро- и макро- моделей.

Отметим достоинства и недостатки вышеизложенных типов моделей. 
Микроскопические модели более вычислительно сложны ввиду необходимости расчётов движения каждого отдельного транспортного средства, однако позволяют более детально просчитывать перекрестки и проводить эксперименты с светофорным управлением въездами. 
Макроскопические модели основанные на гидродинамике наследуют их проблемы с существенно разрывными потоками возникающими при моделировании светофорного управления в транспортных сетях, однако более вычислительно мощные и позволяют моделировать поведение автомобилей на магистралях большой протяженности при любой плотности АТС на ней.
В мезоскопических моделях необходимо рассматривать подход каждого автора в отдельности так как всё зависит от принципов построения изложенной модели.

Моделирование же крупных транспортных сетей на сегодняшний день представлено в~\autocite{asano2015traffic, bieker2015traffic} в
виде примеров применения существующих программных пакетов, таких как SUMO (Simulation of Urban Mobility),
iTETRIS (An Integrated Wireless and Traffic Platform for Real-Time Road Traffic Management Solutions) и др.
Хотя детальное описание подхода к моделированию автомагистрали в данных пакетах зачастую отсутствует, большая часть таких программных пакетов использует микроскопический подход к моделированию транспортной сети~\autocite{ratrout2009comparative} ввиду простоты программной реализации таких подходов и необходимости в моделирования городской транспортной сети с большим числом перекрестков и светофоров.

В данной работе предлагается новый мезоскопический подход к моделированию транспортных потоков в транспортной сети высокой загруженности основанный на моделировании движения не каждого отдельного автомобиля, а движения групп автомобильно-транспортных средств. 
Для расчёта скорости каждой группы в каждый момент времени предлагается использовать фундаментальную диаграмму поток-плотность~\autocite{collectiveArticle2}.
Построение фундаментальной диаграммы на каждом сегменте транспортной сети производится на основе комплексирования данных с дорожных датчиков и GPS-треков.


\ifsynopsis

\else

\fi

% {\progress}
% Этот раздел должен быть отдельным структурным элементом по
% ГОСТ, но он, как правило, включается в описание актуальности
% темы. Нужен он отдельным структурынм элемементом или нет ---
% смотрите другие диссертации вашего совета, скорее всего не нужен.

{\aim} данной работы является разработка мезоскопической модели транспортных потоков на основе комплексированных данных с дорожных датчиков и GPS-треков пригодной для моделирования транспортной сети, а также проверка гипотизы о возможности повышения пропускной способности автомагистрали при полном управлении её въездами с помощью светофоров на основе модели.

Для~достижения поставленной цели необходимо было решить следующие {\tasks}:
\begin{enumerate}[beginpenalty=10000] % https://tex.stackexchange.com/a/476052/104425
  \item Комплексирование данных с дорожных датчиков и GPS-треков
  \item Построение фундаментальной диаграммы поток-плотность для всех сегментов транспортной сети на основе комплексированных данных
  \item Создать мезоскопическую модель транспортных потоков и показать ее состоятельность
  \item Провести моделирование существующей автомагистрали и показать теоретическую возможность повышения ее пропускной способности с помощью адаптивного управления
\end{enumerate}


{\novelty}
\begin{enumerate}[beginpenalty=10000] % https://tex.stackexchange.com/a/476052/104425
  \item Впервые была построена мезоскопическая модель на основе групп АТС с использованием фундаментальной диаграммы поток-плотность на основе комплексированных данных
  \item Было выполнено оригинальное исследование о применимости предложенной модели к адаптивному управлению выделенной автомагистрали с целью потенциального увеличения её пропускной способности
\end{enumerate}

{\influence} Разработанная модель позволяет проводить моделирование транспортной сети любой сложности с использование разнородных источников данных с онлайн-управлением светофорным движением транспортными средствами в сети. 

{\methods} Экспериментальное исследование проводилось с использованием программно-алгоритмического комплекса, разработанного автором.

{\defpositions}
\begin{enumerate}[beginpenalty=10000] % https://tex.stackexchange.com/a/476052/104425
  \item Мезоскопическая математическая модель транспортных потоков на основе групп АТС с использованием фундаментальной диаграммы поток-плотность для расчёта скорости групп на основе комплексированных данных
  \item Второе положение
  \item Третье положение
  \item Четвертое положение
\end{enumerate}


{\reliability} полученных результатов обеспечивается математической точностью изложенных алгоритмов и описаниями проведённых экспериментов, допускающими их воспроизводимость. 
Результаты находятся в соответствии с результатами, полученными другими авторами.


{\probation}
Основные результаты работы докладывались~на:
\begin{enumerate}
  \item 11-я Международная конференция «Интеллектуализация обработки информации», 2016
  \item 18-я Всероссийская конференция с международным участием «Математические методы распознавания образов», 2017
  \item 19-я Всероссийская конференция с международным участием «Математические методы распознавания образов», 2019
  \item XXVII Международной конференции студентов, аспирантов и молодых учёных «Ломоносов», 2020
  \item 13-я Международная конференция «Интеллектуализация обработки информации», 2020
  \item XXVIII Международной конференции студентов, аспирантов и молодых учёных «Ломоносов», 2020
  \item 20-я Всероссийская конференция с международным участием «Математические методы распознавания образов», 2021
\end{enumerate}

{\contribution} Автор принимал активное участие \ldots

\ifnumequal{\value{bibliosel}}{0}
{%%% Встроенная реализация с загрузкой файла через движок bibtex8. (При желании, внутри можно использовать обычные ссылки, наподобие `\cite{vakbib1,vakbib2}`).
    {\publications} Основные результаты по теме диссертации изложены
    в~XX~печатных изданиях,
    X из которых изданы в журналах, рекомендованных ВАК,
    X "--- в тезисах докладов.
}%
{%%% Реализация пакетом biblatex через движок biber
    \begin{refsection}[bl-author, bl-registered]
        % Это refsection=1.
        % Процитированные здесь работы:
        %  * подсчитываются, для автоматического составления фразы "Основные результаты ..."
        %  * попадают в авторскую библиографию, при usefootcite==0 и стиле `\insertbiblioauthor` или `\insertbiblioauthorgrouped`
        %  * нумеруются там в зависимости от порядка команд `\printbibliography` в этом разделе.
        %  * при использовании `\insertbiblioauthorgrouped`, порядок команд `\printbibliography` в нём должен быть тем же (см. biblio/biblatex.tex)
        %
        % Невидимый библиографический список для подсчёта количества публикаций:
        \printbibliography[heading=nobibheading, section=1, env=countauthorvak,          keyword=biblioauthorvak]%
        \printbibliography[heading=nobibheading, section=1, env=countauthorwos,          keyword=biblioauthorwos]%
        \printbibliography[heading=nobibheading, section=1, env=countauthorscopus,       keyword=biblioauthorscopus]%
        \printbibliography[heading=nobibheading, section=1, env=countauthorconf,         keyword=biblioauthorconf]%
        \printbibliography[heading=nobibheading, section=1, env=countauthorother,        keyword=biblioauthorother]%
        \printbibliography[heading=nobibheading, section=1, env=countregistered,         keyword=biblioregistered]%
        \printbibliography[heading=nobibheading, section=1, env=countauthorpatent,       keyword=biblioauthorpatent]%
        \printbibliography[heading=nobibheading, section=1, env=countauthorprogram,      keyword=biblioauthorprogram]%
        \printbibliography[heading=nobibheading, section=1, env=countauthor,             keyword=biblioauthor]%
        \printbibliography[heading=nobibheading, section=1, env=countauthorvakscopuswos, filter=vakscopuswos]%
        \printbibliography[heading=nobibheading, section=1, env=countauthorscopuswos,    filter=scopuswos]%
        %
        \nocite{*}%
        %
        {\publications} Основные результаты по теме диссертации изложены в~\arabic{citeauthor}~печатных изданиях,
        \arabic{citeauthorvak} из которых изданы в журналах, рекомендованных ВАК\sloppy%
        \ifnum \value{citeauthorscopuswos}>0%
            , \arabic{citeauthorscopuswos} "--- в~периодических научных журналах, индексируемых Web of~Science и Scopus\sloppy%
        \fi%
        \ifnum \value{citeauthorconf}>0%
            , \arabic{citeauthorconf} "--- в~тезисах докладов.
        \else%
            .
        \fi%
        \ifnum \value{citeregistered}=1%
            \ifnum \value{citeauthorpatent}=1%
                Зарегистрирован \arabic{citeauthorpatent} патент.
            \fi%
            \ifnum \value{citeauthorprogram}=1%
                Зарегистрирована \arabic{citeauthorprogram} программа для ЭВМ.
            \fi%
        \fi%
        \ifnum \value{citeregistered}>1%
            Зарегистрированы\ %
            \ifnum \value{citeauthorpatent}>0%
            \formbytotal{citeauthorpatent}{патент}{}{а}{}\sloppy%
            \ifnum \value{citeauthorprogram}=0 . \else \ и~\fi%
            \fi%
            \ifnum \value{citeauthorprogram}>0%
            \formbytotal{citeauthorprogram}{программ}{а}{ы}{} для ЭВМ.
            \fi%
        \fi%
        % К публикациям, в которых излагаются основные научные результаты диссертации на соискание учёной
        % степени, в рецензируемых изданиях приравниваются патенты на изобретения, патенты (свидетельства) на
        % полезную модель, патенты на промышленный образец, патенты на селекционные достижения, свидетельства
        % на программу для электронных вычислительных машин, базу данных, топологию интегральных микросхем,
        % зарегистрированные в установленном порядке.(в ред. Постановления Правительства РФ от 21.04.2016 N 335)
    \end{refsection}%
    \begin{refsection}[bl-author, bl-registered]
        % Это refsection=2.
        % Процитированные здесь работы:
        %  * попадают в авторскую библиографию, при usefootcite==0 и стиле `\insertbiblioauthorimportant`.
        %  * ни на что не влияют в противном случае
        \nocite{vakbib2}%vak
        \nocite{patbib1}%patent
        \nocite{progbib1}%program
        \nocite{bib1}%other
        \nocite{confbib1}%conf
    \end{refsection}%
        %
        % Всё, что вне этих двух refsection, это refsection=0,
        %  * для диссертации - это нормальные ссылки, попадающие в обычную библиографию
        %  * для автореферата:
        %     * при usefootcite==0, ссылка корректно сработает только для источника из `external.bib`. Для своих работ --- напечатает "[0]" (и даже Warning не вылезет).
        %     * при usefootcite==1, ссылка сработает нормально. В авторской библиографии будут только процитированные в refsection=0 работы.
}

При использовании пакета \verb!biblatex! будут подсчитаны все работы, добавленные
в файл \verb!biblio/author.bib!. Для правильного подсчёта работ в~различных
системах цитирования требуется использовать поля:
\begin{itemize}
        \item \texttt{authorvak} если публикация индексирована ВАК,
        \item \texttt{authorscopus} если публикация индексирована Scopus,
        \item \texttt{authorwos} если публикация индексирована Web of Science,
        \item \texttt{authorconf} для докладов конференций,
        \item \texttt{authorpatent} для патентов,
        \item \texttt{authorprogram} для зарегистрированных программ для ЭВМ,
        \item \texttt{authorother} для других публикаций.
\end{itemize}
Для подсчёта используются счётчики:
\begin{itemize}
        \item \texttt{citeauthorvak} для работ, индексируемых ВАК,
        \item \texttt{citeauthorscopus} для работ, индексируемых Scopus,
        \item \texttt{citeauthorwos} для работ, индексируемых Web of Science,
        \item \texttt{citeauthorvakscopuswos} для работ, индексируемых одной из трёх баз,
        \item \texttt{citeauthorscopuswos} для работ, индексируемых Scopus или Web of~Science,
        \item \texttt{citeauthorconf} для докладов на конференциях,
        \item \texttt{citeauthorother} для остальных работ,
        \item \texttt{citeauthorpatent} для патентов,
        \item \texttt{citeauthorprogram} для зарегистрированных программ для ЭВМ,
        \item \texttt{citeauthor} для суммарного количества работ.
\end{itemize}
% Счётчик \texttt{citeexternal} используется для подсчёта процитированных публикаций;
% \texttt{citeregistered} "--- для подсчёта суммарного количества патентов и программ для ЭВМ.

Для добавления в список публикаций автора работ, которые не были процитированы в
автореферате, требуется их~перечислить с использованием команды \verb!\nocite! в
\verb!Synopsis/content.tex!.
