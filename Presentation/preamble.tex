\begin{frame}[noframenumbering,plain]
    \setcounter{framenumber}{1}
    \maketitle
\end{frame}

\begin{frame}
    \frametitle{Положения, выносимые на защиту}
    \begin{itemize}
        \item Мезоскопическая математическая модель транспортных потоков на основе групп АТС с использованием фундаментальной диаграммы поток-плотность для расчёта скорости групп на основе комплексированных данных
        \item Алгоритм комплексирования данных
        \item Результаты вычислительного эксперимента по адаптивному управлению въездами на МКАД на основе созданной мезоскопической модели
        \item Сравнение результатов и скорости вычислений с моделью разумного водителя (IDM)
    \end{itemize}
\end{frame}
\note{
    Проговариваются вслух положения, выносимые на защиту
}

\begin{frame}
    \frametitle{Содержание}
    \tableofcontents
\end{frame}
\note{
    Работа состоит из восьми глав.

    \medskip
    В первой главе приводятся основные подходы к математическому моделированию транспортных потоков.

    Во второй главе рассматривается процедура построения фундаментальной диаграммы поток-плотность на выбранном участке автомагистрали.
    
    В третьей главе рассматривается задача восстановления данных GPS-треков с помощью данных с дорожных датчиков.

    Четвертая глава посвящена математическому описанию предлагаемой модели.

    В пятой главе проводятся вычислительные эксперименты с целью проверки работоспособности модели.

    В шестой главе проводится моделирование МКАД с использованием нескольких фундаментальных диаграмм.

    В седьмой главе проводится моделирование МКАД с расчётом фундаментальных диаграмм для каждого сегмента автомагистрали.
    
    В восьмой главе проводится сравнение результатов и скорости моделирования предложенной моделью и моделью разумного водителя.
}
