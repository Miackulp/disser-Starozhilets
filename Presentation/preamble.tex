\begin{frame}[noframenumbering,plain]
    \setcounter{framenumber}{1}
    \maketitle
\end{frame}

\begin{frame}
    \frametitle{Положения, выносимые на защиту}
    \begin{itemize}
        \item Мезоскопическая математическая модель транспортных потоков на основе групп АТС с использованием фундаментальной диаграммы поток-плотность для расчёта скорости групп на основе комплексированных данных
        \item Алгоритм комплексирования данных
        \item Результаты вычислительного эксперимента по адаптивному управлению въездами на МКАД на основе созданной мезоскопической модели
    \end{itemize}
\end{frame}
\note{
    Проговариваются вслух положения, выносимые на защиту
}

\begin{frame}
    \frametitle{Содержание}
    \tableofcontents
\end{frame}
\note{
    Работа состоит из шести глав.

    \medskip
    В первой главе приводятся основные подходы к математическому моделированию транспортных потоков.

    Во второй главе рассматривается процедура построения фундаментальной диаграммы поток-плотность на выбранном участке автомагистрали.

    Третья глава посвящена математическому описанию предлагаемой модели.

    В четвёртой главе проводятся вычислительные эксперименты с целью проверки работоспособности модели.

    В пятой главе проводится моделирование МКАД с использованием нескольких фундаментальных диаграмм.

    В шестой главе проводится моделирование МКАД с расчётом фундаментальных диаграмм для каждого сегмента автомагистрали.
}
